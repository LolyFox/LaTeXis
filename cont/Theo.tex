\subsection{Méthode par calcul théorique en utilisant le PFD}
\subsubsection{Relation relatives au moteur}
\subsubsection{Équations relatives au réducteur (supposé sans pertes)}

\subsection{diagramme PERT}

\begin{figure}[h!]
	\centering
	
	\tikzset{nodePERT/.style={draw,circle, append after command={
				[shorten >=\pgflinewidth, shorten <=\pgflinewidth,]
				([yshift=0.2mm]\tikzlastnode.base) edge  ([yshift=0.4mm]\tikzlastnode.south)
				([xshift=-0.4mm]\tikzlastnode.east) edge  ([xshift=0.4mm]\tikzlastnode.west)
			}
		}
	}
	\tikzset{line/.style={very thick,->,>=stealth'}}
	\tikzset{radius/.style={very thick,-,>=stealth'}}
	
	\tikzset{figtive/.style={->,>=stealth',thick,dashed}}
	\tikzset{cp/.style={double,double equal sign distance,-implies}}
	
	
	\begin{tikzpicture}[thick,scale=1.15, every node/.style={transform shape}]
	%node 0
	\node (nodePERT-aa)[nodePERT, fill=white, scale=4] {};
	\node[above=-0.65cm] at (nodePERT-aa.north) {0};
	\node[below left] at (nodePERT-aa) {0};
	\node[below right] at (nodePERT-aa) {0};
	
	%node 1
	\node (nodePERT-bb)[nodePERT,right of=nodePERT-aa,above=3cm ,node distance=4cm, scale=4] {};
	\node[above=-0.65cm] at (nodePERT-bb.north) {1};
	\node[below left] at (nodePERT-bb) {6};
	\node[below right] at (nodePERT-bb) {9};
	
	%node 2
	\node (nodePERT-cc)[nodePERT,right of=nodePERT-aa,above=1cm ,node distance=3cm, fill=white, scale=4] {};
	\node[above=-0.65cm] at (nodePERT-cc.north) {2};
	\node[below left] at (nodePERT-cc) {9};
	\node[below right] at (nodePERT-cc) {12};
	
	%node 3
	\node (nodePERT-dd)[nodePERT,right of=nodePERT-aa,below=1cm ,node distance=3cm, fill=white, scale=4] {};
	\node[above=-0.65cm] at (nodePERT-dd.north) {3};
	\node[below left] at (nodePERT-dd) {4};
	\node[below right] at (nodePERT-dd) {12};
	
	%node 4
	\node (nodePERT-ee)[nodePERT,right of=nodePERT-aa,below=3cm ,node distance=4cm, fill=white, scale=4] {};
	\node[above=-0.65cm] at (nodePERT-ee.north) {4};
	\node[below left] at (nodePERT-ee) {7};
	\node[below right] at (nodePERT-ee) {7};
	
	%node 5
	\node (nodePERT-ff)[nodePERT,right of=nodePERT-aa ,node distance=6.5cm, fill=white, scale=4] {};
	\node[above=-0.65cm] at (nodePERT-ff.north) {5};
	\node[below left] at (nodePERT-ff) {12};
	\node[below right] at (nodePERT-ff) {12};
	
	%node 6
	\node (nodePERT-gg)[nodePERT,right of=nodePERT-aa,above=3cm ,node distance=9cm, fill=white, scale=4] {};
	\node[above=-0.65cm] at (nodePERT-gg.north) {6};
	\node[below left] at (nodePERT-gg) {17};
	\node[below right] at (nodePERT-gg) {18};
	
	%node 7
	\node (nodePERT-hh)[nodePERT,right of=nodePERT-aa,below=3cm ,node distance=9cm, fill=white, scale=4] {};
	\node[above=-0.65cm] at (nodePERT-hh.north) {7};
	\node[below left] at (nodePERT-hh) {16};
	\node[below right] at (nodePERT-hh) {16};
	
	%node 8
	\node (nodePERT-ii)[nodePERT,right of=nodePERT-aa ,node distance=12cm, fill=white, scale=4] {};
	\node[above=-0.65cm] at (nodePERT-ii.north) {8};
	\node[below left] at (nodePERT-ii) {21};
	\node[below right] at (nodePERT-ii) {21};
	
	% coodonate 1
	\coordinate [right of=nodePERT-aa,above=3.7cm ,node distance=1.5cm] (coordonate1);
	
	% coodonate 2
	\coordinate [right of=nodePERT-aa,above=1.7cm ,node distance=1.25cm] (coordonate2);
	
	% coodonate 3
	\coordinate [right of=nodePERT-aa,below=1.7cm ,node distance=1.25cm] (coordonate3);
	
	% coodonate 4
	\coordinate [right of=nodePERT-aa,below=3.7cm ,node distance=1.5cm] (coordonate4);
	
	% A' task
	\draw[line] (nodePERT-aa) -- (coordonate1) -- node [above] {$A'$} node [below] {6} (nodePERT-bb);
	
	% B task
	\draw[line] (nodePERT-aa) -- (coordonate2) --  node [above] {$B$} node [below] {9} (nodePERT-cc);
	
	% C task
	\draw[line] (nodePERT-aa) --  node [above] {$C$} node [below] {11} (nodePERT-ff);
	
	% D task
	\draw[line] (nodePERT-aa) -- (coordonate3) -- node [above] {$D$} node [below] {4} (nodePERT-dd);
	
	% E' task
	\draw[cp] (nodePERT-aa) -- (coordonate4) -- node [above] {$E'$} node [below] {7} (nodePERT-ee);
	
	
	% A'' task
	\draw[line] (nodePERT-bb) --  node [above] {$A''$} node [below] {4} (nodePERT-gg);
	
	% E'' task
	\draw[line] (nodePERT-ee) --  node [above] {$E''$} node [below] {7} (nodePERT-hh);
	
	
	% figtive1 task
	\draw[figtive] (nodePERT-cc) --  node [below] {$\varnothing$}  (nodePERT-ff);
	
	% figtive2 task
	\draw[figtive] (nodePERT-dd) --  node [above] {$\varnothing$} (nodePERT-ff);
	
	
	
	% F task
	\draw[line] (nodePERT-bb) --  node [above right] {$F$} node [below left] {3} (nodePERT-ff);
	
	% G task
	\draw[cp] (nodePERT-ee) --  node [above left] {$G$} node [below right] {5} (nodePERT-ff);
	
	
	% J task
	\draw[line] (nodePERT-ff) --  node [above] {$J$} node [below] {6} (nodePERT-ii);
	
	% H task
	\draw[line] (nodePERT-ff) --  node [above left] {$H$} node [below right] {5} (nodePERT-gg);
	
	% K task
	\draw[cp] (nodePERT-ff) --  node [above right] {$K$} node [below left] {4} (nodePERT-hh);
	
	% I task
	\draw[line] (nodePERT-gg) --  node [above right] {$I$} node [below left] {3} (nodePERT-ii);
	
	% L task
	\draw[cp] (nodePERT-hh) --  node [above left] {$L$} node [below right] {5} (nodePERT-ii);
	
	\end{tikzpicture}
	\caption{Diagramme PERT du projet EADS}
\end{figure}

\newpage
\subsection{TP d'automatique}

\begin{figure}[h!]
	\centering
	\begin{tikzpicture}
	
	\pgfmathdeclarefunction{p}{1}{%
		\pgfmathparse{(and(#1>-0.2, #1<10))}%
	}
	
	\begin{axis}[title={\sffamily Réponse indicielle d'un système du second ordre $0 < \zeta < 1$},  
	height=0.3\textheight,
	width=0.85\textwidth,
	xlabel={$t$},
	ylabel={$y_1(t)$},
	xtick={0, 0.5, ..., 3},
	ytick={0, 0.2, ..., 1.2},
	xmin=0, xmax=3,
	ymin=0, ymax=1.2,
	domain=0:3,
	grid=major, % Display a grid
	grid style={dotted,gray!50}, % set the style
	legend style={draw=none,at={(0.75,0.25)},anchor=north},
	legend cell align=left
	]
	
	\addplot[dashed]{1};
	
	\addlegendentry{Asymptote}
	
	
	\addplot[color=red,smooth,
	thick,
	mark=none,
	samples=500,domain=0:2*pi]{
		%(1-(e^(-0.5*5*x)/sqrt(1-(0.5)^2) * (\sin(sqrt(1-(0.5)^2)*5*x)*0.5+\cos(sqrt(1-(0.5)^2)*5*x)*sqrt(1-(0.5)^2)) ) 
		(1-(e^(-0.5*5*x)/sqrt(1-(0.5)^2) *(sin(sqrt(1-(0.5)^2)*5*deg(x))*0.5+cos(sqrt(1-(0.5)^2)*5*deg(x))*sqrt(1-(0.5)^2))
	};
	\addlegendentry{Réponse indicielle $y_1(t)$}
	
	\draw [densely dashed] ({rel axis cs:0,0} -| {axis cs:1.45,0})  -- ({rel axis cs:0,0.81} -| {axis cs:1.45,0})  ;	
	\draw [densely dashed] ({rel axis cs:0,0} -| {axis cs:0.72,0}) -- ({rel axis cs:0,0.97} -| {axis cs:0.72,0})  ;	
	
	\end{axis}
	
	\node[] at (2.879, 0) (a) {$\bullet$};
	\node[below left=0cm and -0.60cm of a]{$t_m$};
	
	\node[] at (5.779, 0) (b) {$\bullet$};
	\node[above left=0cm and 0cm of b]{$t_a$};
	\end{tikzpicture}
	\caption{\sffamily Réponse indicielle d'un système du second ordre $0 < \zeta < 1$}
\end{figure}